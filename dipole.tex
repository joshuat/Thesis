\section{Optical Dipole Trapping}

The use of optical dipole trapping in cold-atom electron sources will allow greater stability of the atom cloud during ionisation and extraction as well as increasing the density of the atom cloud during these phases with corresponding increases in density.

\subsection{History of dipole trapping}
The use of the optical dipole force as a confining mechanism was first proposed by Askar'yan in 1962\cite{askar'yan_effects_1962} for plasmas and neutral atoms. Ashkin successfully deomstarted the trapping of micron-size latex spheres suspended in water using a focussed guassian lasers in 1970\cite{ashkin_acceleration_1970}. The first optical trapping of atoms was demonstrated by Chu et. al. in 1986\cite{chu_experimental_1986} where a optical dipole trap was used to trap sodium atoms.

Since then optical dipole traps have been used extensively for such things as all-optical bose-einstein condensation\cite{barrett_all-optical_2001}.

****Any other modern stuff worth mentioning?****

\subsection{Theory of Dipole Trapping}

\subsection{How we'll use a dipole trap}
That title is terrible by the way.

\subsubsection{Wavelength}
`Close' to 780nm vs. 1064nm

\subsubsection{Configurations}
single axis vs. crossed

\subsubsection{Waist size}

\subsection{Optical lattice}
Dipole trap used as initial trap

\subsection{All optical trapping}
