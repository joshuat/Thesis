\chapter{Introduction}
\pagenumbering{arabic} % This works correctly if it's here

The University of Melbourne's \gls{caes} project aims to create high-brightness,  high-coherence electron bunches for use in coherent electron diffractive imaging. The imaging of nanoscale objects such as biological molecules \cite{dwyer_femtosecond_2006, williamson_clocking_1997} and defects in solid-state devices \cite{siwick_atomic-level_2003} by ultrafast, single-shot electron diffractive imaging could provide important information about structures and dynamic processes of important biological molecules such as membrane proteins.

\section{Imaging of Bio-molecules}

In order to determine the structure of biological molecules imaging techniques with atomics resolution are required. A number of techniques are capable of determining these structures \cite{nettleship_methods_2008, svergun_small-angle_2003, opella_structure_2004} and the most successful to date has been x-ray crystallography \cite{kendrew_three-dimensional_1958, uson_advances_1999}. Unfortunately the crystallisation process is difficult and to date only a small proportion of proteins have been successfully crystallised \cite{geerlof_impact_2006}.

\subsection{Membrane Proteins}
Membrane proteins are the proteins that are associated with or attached to the membranes of cells. They are involved in detecting and conveying external signals into cells. This allows the cells to interact with and respond to their environment\cite{almen_mapping_2009}. Membrane proteins are important in determining immune responses, interactions with pharmaceuticals, cell adhesion to form tissues and in controlling important metabolic processes such as salt balance, energy production and photosynthesis\cite{chiras_human_2011}.

Determining the structure of theses molecules is a key step in understanding their chemical and biological function. The importance of knowing the atomic structure of biomolecules is exemplified by the enormous progress made in various fields of biology once the double-helical structure of DNA was determined from x-ray images in 1953 \cite{watson_molecular_1953}. Once a protein's structure and function are known then it becomes possible to design drugs \cite{pinto_influenza_1992} where needed and to more fully understand how the protein behaves in its biological system.

The developement of new imaging techniques, such as ultrafast single shot diffraction, and new light sources, such as \glspl{xfel}, have been driven by the goal of overcoming the limitations of x-ray crystallography. Ultrafast single-shot diffraction imaging also has the potential to able to determine the dynamic structure of biological molecules.



\subsection{Ultrafast, single-shot, coherent diffractive imaging with electrons}

X-ray diffraction from crystals was first observed a century ago\cite{bragg_x-rays_1912} and resulted in a Nobel prize being awarded to William Bragg and his son. Since then \gls{cdi} has been performed on a myriad of different samples with coherent beams of x-rays and electrons.

Ultrafast, single-shot imaging requires a very bright source of radiation in order to provide enough scattered information on the detector before the sample takes damage from the illumination\cite{henderson_potential_1995}. Single-shot imaging with a sufficiently bright source and a short enough interaction time would be able to image protein molecules, and other bio-molecules, without the need for crystallisation\cite{neutze_potential_2000}.

With femtosecond timescale single-shot imaging it is possible to observe such things as molecular vibration and dynamic chemical processes\cite{zewail_4d_2006}. With the sophisticated imaging techniques currently in development and the continued improvement of radiation sources it will become possible to create `molecular movies'\cite{dwyer_femtosecond_2006} of these processes.

Unfortunately, the brightnesses required for single shot imaging requires multi-billion-dollar \gls{xfel} facilities. Electron sources however may be able to provide cheaper alternatives. Electron interactions with molecules are significantly stronger than those of x-rays. For similar energies the electron interaction with a sample is $10^5-10^6$ times stronger than that of x-rays\cite{sciaini_femtosecond_2011}.

\section{Melbourne cold-atom electron source}

The University of Melbourne's \gls{caes} project aims to produce an electron source for coherent diffractive imaging. If bright, coherent, femtosecond long bunches of electrons can be produced then \gls{cdi} can be performed on a range of structures and eventually molecules.

In order to produce the electrons Rubidium atoms are cooled and trapped in a retro-reflective \gls{mot} that is loaded from a Zeeman slower\cite{phillips_laser_1982, phillips_cooling_1987, bell_slow_2010}. The valance electrons of the trapped atoms are ionised using a red (780nm) and a blue (480nm) laser.

The wavelength of the blue ionisation laser is tuned such that the electrons are given the minimum energy to ionise ensuring that the electrons are `cold'.

These cold electrons are then accelerated out of the cloud by a uniform electric field, guided to the neighbouring sample chamber and focused onto the target sample. After passing through the sample the resulting diffraction pattern is incident upon the detector.

The diffraction pattern recorded by the detector can be used to determine the structure of the sample using standard diffraction techniques.

\section{Current issues with cold-atom electron source}

The \gls{caes} is not perfect. :(

\begin{itemize}
    \item stability
    \item density
\end{itemize}

\section{Intro to dipole traps}

    \subsection{How the ODT will fix all woes}


