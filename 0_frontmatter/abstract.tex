\chapter*{\centering \LARGE Abstract}
\begin{quotation}
\noindent

The University of Melbourne's \gls{caes} project aims to create short, high-brightness, high-coherence electron bunches for use in ultra-fast, single-shot \gls{cdi}. The imaging of nanoscale objects such as biological molecules and defects in solid-state devices by ultrafast, single-shot electron diffractive imaging would provide important information about structure and dynamic processes. If the \gls{caes} is successful then it will be able to determine the structure of essential molecules, such as membrane proteins, and create `molecular movies' of chemical reactions.

The \gls{caes} operates by ionising rubidium-85 atoms that have been cooled and trapped in a \gls{mot}. This ionisation is a two-stage process that produces extremely cold electron bunches ($T=10\,\unit{K}$) which are accelerated towards the sample under investigation. During ionisation the trapping must be turned off. This causes the atom cloud to expand and residual magnetic fields interfere with the trajectories of the electron bunches.

An \gls{odt} has been designed, constructed and integrated with the \gls{caes} during the course of this project. The \gls{odt} stabilises the trajectories of the electron bunches which is essential for imaging. The \gls{odt} consists of two $700\,\unit{mW}$ Gaussian laser beams whose foci overlap with each other and the \gls{mot} atom cloud. Atoms feel a force towards the high intensity regions of the \gls{odt} beams, such as the foci. This serves to trap the atoms and is used to hold the atoms in place while the residual magnetic fields from the \gls{mot} dissipate and thus produce electron bunches free from magnetic disturbance.

\end{quotation}
\clearpage
