\chapter{Results}

To obtain high quality electron diffraction the number of atoms available needs to be maximised and they need to be located in the smalled possible volume with a highly stable distribution. In order to achieve these goals with the \gls{odt} it is first necessary to establish a high density \gls{mot} cloud, with temperature of order 10 times the \gls{odt} depth. \Gls{mot} optimisation is beyond the scope of this project but nethertheless it is important to measure the \gls{mot} characteristics in order to be able to optimise the final characteristics of the \gls{odt}. Measurements of the \gls{odt} are necessary for the analysis and optimisation of the \gls{odt} parameters. The results of the \gls{mot} and \gls{odt} measurements are described here.

\section{Magneto-Optic Trap Analysis}
Analysis of the \gls{mot} allows us to optimise the \gls{mot} characteristics and the \gls{odt} parameters to have the best performance for the \gls{caes}. The temperature and number of trapped atoms in the \gls{mot} were calculated as shown below.

\subsection{Temperature}
Knowing the temperature of the \gls{mot} allows optimisation of the parameters of the \gls{odt} and can be useful for optimising the \gls{mot} itself.

The procedure for calculating the temperature was discussed in section \ref{temp_measurement}. The data for this measurement was collected by releasing the \gls{mot}, waiting for a number of milliseconds and then imaging the cloud.

A program was created to perform this calculation\footnote{Written in Python, approximately 400 lines long}. The data is first read from it's binary storage format into the program. The data was rotated in order to align the long and short axes of the ellipticallly shaped \gls{mot} cloud with the $x$ and $y$ axes. A two dimentional, elliptical Gaussian distribution was then fitted to each of these images (see figure \ref{fig:mot_example_images} for examples). The 1/e radius of each of these distributions was used to fit to equation \ref{eq:cloud_radius} and the temperature was calculated with equation \ref{eq:temp_velocity}. The separate values for $x$ and $y$ were then averaged. The errors from the Gaussian fits combined with the error from the fit to equation\ref{eq:cloud_radius} were used to calculate the uncertainty. The results of this program are given below and the fitting is shown in figure \ref{fig:temp_fits}.

\begin{figure}[h]
\begin{subfigure}[b]{0.5\textwidth}
    \includegraphics[width=\textwidth]{figs/final_temp_fitting_x.png}
\end{subfigure}\begin{subfigure}[b]{0.5\textwidth}
    \includegraphics[width=\textwidth]{figs/final_temp_fitting_y.png}
\end{subfigure}
\caption{Measurements of thermal expansion of the MOT (red dots) and the fitted curve (red line).}
\label{fig:temp_fits}
\end{figure}

The temperature of the \gls{mot} was determined to be $139\pm22\,\unit{\mu K}$. Since the previous measurements of $70\,\unit{\mu K}$\cite{sheludko_shaped_2010} the detuning of the \gls{mot} cooling beams has been increased to optimise the loading rate and the fact that temperature scales with the detuning explains this increase.

\subsection{Atom Count}
\label{mot_atom_count}
The number of atoms in the \gls{mot} is a critical metric as this determines the number of atoms available for loading into the \gls{odt} and thus for ionisation. The theory involved in determining the atom number is discussed in section \ref{atom_count}. The imaging laser intensity is well above saturation in the centre beam and thus equation \ref{eq:atom_count} must be used.

Another program was created to perform this calculation\footnote{Written in Python, approximately 200 lines long.} and data used for the temperature measurements was analysed. The number of atoms counted remains fairly constant until the cloud reaches the limits of the \gls{ccd} field of view after which there is a steep decline in the atom count (\ref{fig:mot_atom_count}). The number of atoms in the \gls{mot} is therefore $1.3 \pm0.2\times10^8$ which is comparable to measurements made previously with the \gls{caes}\cite{sheludko_shaped_2010}.

\begin{figure}[t]
\centering
    \begin{subfigure}[b]{0.3\textwidth}
    \centering
\begin{tikzpicture}
    \node[anchor=south west,inner sep=0] (image) at (0,0) {\includegraphics[width=0.8\textwidth]{figs/MOTimage1.png}};
    \begin{scope}[x={(image.south east)},y={(image.north west)}]
        \draw[black,thick] (0.2,0.15) -- node[above]{$5\,\unit{mm}$} (0.49524,0.15);
    \end{scope}
\end{tikzpicture}
\begin{tikzpicture}
\node[anchor=south west,inner sep=0] (image) at (0,0) {\includegraphics[width=0.8\textwidth]{figs/MOTimage2.png}};
    \begin{scope}[x={(image.south east)},y={(image.north west)}]
        \draw[black,thick] (0.2,0.15) -- node[above]{$5\,\unit{mm}$} (0.49524,0.15);
    \end{scope}
\end{tikzpicture}

    \caption{False colour absorption images of the MOT $1\,\unit{ms}$ (top) and $9\,\unit{ms}$ (bottom) after release.}
    \label{fig:mot_example_images}
    \end{subfigure}~~~\begin{subfigure}[b]{0.6\textwidth}
    \centering
    \includegraphics[width=0.8\textwidth]{figs/MOT_atom_count.pdf}
    \caption{The evolution of the number of atoms in the MOT cloud imaged by the camera during themal expansion.}
    \label{fig:mot_atom_count}
    \end{subfigure}
    \caption{}
\end{figure}

\section{Optical Dipole Trap Analysis}
The first optical dipole trapping was observed in September 2012 and the first crossed \gls{odt} images acquired in early October 2012. Initially trapping along only the first beam of the \gls{odt} was observed despite both beams beaing supposedly aligned. After further alignment and changes to the beam focussing trapping was observed in both beams with no evidence of a region of overlapped traps. Eventually however the beams were properly aligned and crossed trapping was observed as shown in figure \ref{fig:ODTimage1}.

An interesting effect can be observed in the crossed \gls{odt} images. Atoms within the \gls{odt} region can be seen being pushed down the beam by the scattering force of the trapping laser while leaving the core of the \gls{odt} behind. This is shown in figure \ref{fig:crossed_effect}. This effect implies that the \gls{odt} is not deep enough to trap all of the atoms within the trapping region. This can be fixed by either increasing the depth of the \gls{odt} or reducing the temperature of the \gls{mot}.

The \gls{odt} absorption images were acquired by turning off the \gls{mot} trapping while the \gls{odt} was on. The atoms that were trapped in the \gls{mot} begin to expand thermally while those that are trapped in the \gls{odt} do not. An absorption image was acquired at certain times after the \gls{mot} trapping had ceased. This scenerio was repeated with the \gls{odt} turned off for the entire process and this image was used as $I_0$ in equation \ref{eq:transmission_function_1}. This process is not perfect however as most of the atoms in the \gls{odt} are also present in the background image. As you can see from the images however this technique is adequate for the initial calculations performed here.

The same set of data was used for all of the analysis performed in this section. In this data set the \gls{odt} laser was tuned to $\lambda=780.317\,\unit{nm}$ and it had only $358\,\unit{mW}$ of power. The \gls{mot} trapping was turned off for a range of times before imaging occured.

\begin{figure}[h]
    \centering
    \begin{subfigure}[b]{0.3\textwidth}
\begin{tikzpicture}
    \node[anchor=south west,inner sep=0] (image) at (0,0) {\includegraphics[width=1\textwidth]{figs/ODTimage1.pdf}};
    \begin{scope}[x={(image.south east)},y={(image.north west)}]
        \draw[white,thick] (0.2,0.15) -- node[above]{$5\,\unit{mm}$} (0.49524,0.15);
    \end{scope}
\end{tikzpicture}
    \end{subfigure}\begin{subfigure}[b]{0.3\textwidth}
\begin{tikzpicture}
    \node[anchor=south west,inner sep=0] (image) at (0,0) {\includegraphics[width=1\textwidth]{figs/ODTimage2.pdf}};
    \begin{scope}[x={(image.south east)},y={(image.north west)}]
        \draw[white,thick] (0.2,0.15) -- node[above]{$5\,\unit{mm}$} (0.49524,0.15);
    \end{scope}
\end{tikzpicture}
    \end{subfigure}\begin{subfigure}[b]{0.3\textwidth}
\begin{tikzpicture}
    \node[anchor=south west,inner sep=0] (image) at (0,0) {\includegraphics[width=1\textwidth]{figs/ODTimage3.pdf}};
    \begin{scope}[x={(image.south east)},y={(image.north west)}]
        \draw[white,thick] (0.2,0.15) -- node[above]{$5\,\unit{mm}$} (0.49524,0.15);
    \end{scope}
\end{tikzpicture}
    \end{subfigure}
\caption{False colour absorption images of a crossed ODT showing untrapped atoms in the region of the ODT being pushed down the trapping beams. The images are taken 2, 4 and $6\,\unit{ms}$ after the MOT trapping has be turned off.}
\label{fig:crossed_effect}
\end{figure}

\subsection{Size}
The size of the atom cloud in the crossed \gls{odt} can be calulated by fitting Gaussian curves to the distributions in the absorption images (figure \ref{fig:ODTimage1}). Taking the 1/e radius from the fit the 1/e length of the trap was found to be $1.3\pm0.15\,\unit{mm}$ and the width to be $0.4\pm0.02\,\unit{mm}$. The \gls{mot} trapping was turned off for $2\,\unit{ms}$ before imaging. Another program was created to do this analysis\footnote{Written in Python, approximately 300 lines long.}.

\begin{figure}[h]
\centering
    % row 1
    \begin{subfigure}[b]{0.5\textwidth}\centering
\begin{tikzpicture}
    \node[anchor=south west,inner sep=0] (image) at (0,0) {\includegraphics[width=0.5\textwidth]{figs/ODTimage1.png}};
    \begin{scope}[x={(image.south east)},y={(image.north west)}]
        \draw[white,thick] (0.2,0.15) -- node[above]{$5\,\unit{mm}$} (0.49524,0.15);
    \end{scope}
\end{tikzpicture}
        \caption{False colour absorption image of the crossed ODT.}
    \end{subfigure}~~~\begin{subfigure}[b]{0.5\textwidth}\centering
\includegraphics[width=0.5\textwidth]{figs/ODTimage1x.png}
        \caption{Vertical cross-section through the centre of the atom cloud in the ODT along with the fitted Gaussian.}
    \end{subfigure}


    % row 2
    \begin{subfigure}[b]{0.5\textwidth}\centering
        \begin{tikzpicture}
    \node[anchor=south west,inner sep=0] (image) at (0,0) {\includegraphics[width=0.6\textwidth]{figs/ODTimage1zoom.png}};
    \begin{scope}[x={(image.south east)},y={(image.north west)}]
        \draw[white,thick] (0.2,0.73) -- node[below left]{\footnotesize $1\,\unit{mm}$} (0.7351,0.73);
    \end{scope}
\end{tikzpicture}
\caption{A close up of the atom cloud shown in a.}
    \end{subfigure}~~~\begin{subfigure}[b]{0.5\textwidth}
        \centering
        \includegraphics[width=0.5\textwidth]{figs/ODTimage1y.png}
        \caption{Horizontal cross-section through the centre of the atom cloud in the ODT along with the fitted Gaussian.}
    \end{subfigure}

    \caption{}
    \label{fig:ODTimage1}
\end{figure}


\subsection{Atom Count}
The number of atoms in the \gls{odt} can be calculated using the same methods and program used in section \ref{mot_atom_count} with slight modifications.

As mentioned above the \gls{mot} cloud is still expanding during the collection of this data so the number of residual atoms must be taken into account. This is done by using the atom-free image as a background ($I_0$) combined with an image of the expanding \gls{mot} ($I_{mot}$) and an image of the expanding \gls{mot} with the \gls{odt} turned on ($I_{mot+odt}$). Both images were acquired the same amount of time after the \gls{mot} was released. This provides us with the number of atoms in the expanding \gls{mot}, $N_{mot}$, and the number of atoms in the expanding \gls{mot} with the \gls{odt} turned on, $N_{mot+odt}$. Thus we can calculate the number of atoms in the \gls{odt}, $N_{odt}=N_{mot+odt}-N_{mot}$. Again this is not a perfect calculation as many of the atoms trapping in the \gls{odt} are taken from the \gls{mot} distribution however it is sufficient as preliminary data.

As shown in figure \ref{fig:lifetime} the number of atoms in the dipole trap at $t=0$ is $6.05\times10^6$ which corresponds to approximately 5\% of the atoms trapped in the \gls{mot}.

\subsubsection{Lifetime}
Measurement of the lifetime of the \gls{odt} are important since the \gls{caes} requires a high number of atoms available for ionisation after the \gls{mot} trapping has been turned off. Ideally the \gls{odt} will have a lifetime well in excess of $10\,\unit{ms}$. The simple model described in section \ref{lifetime_section} has been used to determine the 1/e lifetime of the \gls{odt} using another program\footnote{Written in Python, approximately 150 lines long.}. The lifetime has been calculated to be $2.1\pm0.9\,\unit{ms}$ with the wavelength and power detailed at the start of this section.

\begin{figure}[h]
\centering
\includegraphics[width=0.4\textwidth]{figs/lifetime.pdf}
\caption{Number of atoms in the ODT over time.}
\label{fig:lifetime}
\end{figure}
