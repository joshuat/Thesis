\section{Introduction}

% Motivation
The University of Melbourne's cold-atom electron source aims to be able to create high-brightness, high-coherence electon bunches for use in coherent electron diffractive imaging. Imaging of nanoscale objects such as biological molecules\cite{dwyer_femtosecond_2006, williamson_clocking_1997} and defects in solid-state devices\cite{siwick_atomic-level_2003} by ultrafast, single-shot electron diffractive imaging would provide important information about structure and dynamic processes of nanoscale objects.

Membrane proteins, for example, are very important for some reason \emph{(ASK VIVIEN get some references)}. Determining the structure of these molecules is a key step in understanding their chemical and biological function. The importance of knowing the atomic structure of biomolecules is exemplefied by the enormous progress made in various fields of biology once the double-helical structure of DNA was determined from x-ray images in 1953\cite{watson_molecular_1953}. Once a protein's structure and function are known then it becomes possible to design drugs\cite{pinto_influenza_1992} where needed and to more fully understand how the protein behaves in its biological system.

In order to determine the structure of these biological molecules atomic, sub-nanometre imaging resolution is required. A number of techniques are available for determing these structures \cite{nettleship_methods_2008, svergun_small-angle_2003, opella_structure_2004} however the most successful to date has been x-ray crystallography \cite{kendrew_three-dimensional_1958, uson_advances_1999}. Unfortunately the process of crystallising these membrane proteins is difficult and to date relatively few have been crystallised \cite{geerlof_impact_2006}.

New imaging techniques and light sources such as x-ray free electron lasers and  ultrafast single-shot diffraction have been driven by the goal of overcoming the limitations of x-ray crystallography. Ultrafast single-shot diffraction imaging also has the potential to determine dynamic structure of biological molecules. The Melbourne cold-atom electron source is aims to produce bright, coherent bunches of electrons for use in diffactive imaging.

\subsection{Ultrafast, single-shot, coherent diffractive imaging with electrons}
X-ray diffraction from crystals was first observed a century ago\cite{bragg_x-rays_1912} and resulted in a Nobel prize being awarded to William Bragg and his son. Since then \gls{cdi} has been performed on a myriad of different samples with coherent beams of x-rays and electrons.

Electrons have a shorter wavelength than x-rays thus allowing a higher limit on the attainable resolution for \gls{cdi}. [REFERENCE would be nice] 

\subsubsection{Single-shot diffractive imaging}
Single-shot diffractive imaging with an x-ray source of sufficient brightness should be able to produce a diffaction pattern from scattered x-rays from a single molecule before the molecule is destroyed by the Coulomb explosion which follows photoionisation within the molecule\cite{henderson_potential_1995, neutze_potential_2000}. Single-shot imaging aims to avoid the need for crystallisation with x-ray imaging since with a sufficiently bright source should allow imaging of any molecule.

With femtosecond timescale single-shot imaging it becomes possible to observe such things as molecular vibration and dynamic chemical processes\cite{zewail_4d_2006}.

\subsection{Melbourne cold-atom electron source}
we do... stuff
aim to get
\begin{itemize}
    \item brightness (give definition)
    \item coherence
    \item bunches
\end{itemize}

problems
\begin{itemize}
    \item coulomb explosion (bunchshaping)
    \item bunch length
    \item brightness (dipole trap and other stuff)
\end{itemize}

\subsubsection{Brightness}
The transverse brightness at the source is given by\cite{reiser_theory_2008}
\begin{equation}
B_\perp = \frac{I_p m_e c^2}{4 \pi^2 \sigma_x \sigma_y k_B T}
\end{equation}
where $\sigma$ is the root mean squared source size and $I_p$ is the peak electron current.

A pulse of electron's brightness can be increased by a reduction in the length of the bunch or by increasing the density of the source. A short bunch is necessary for ultrafast electron diffraction.

Increasing the density of the source can be achieved with an optical dipole trap which is the focus of this project.

\subsubsection{Coherence}
For a quasi-homogeneous source\cite{nugent_coherent_2009}, the transverse coherence length $L_c$ can be related to the transverse momentum spread, and hence the temperature, through\cite{van_oudheusden_electron_2007}
\begin{equation}
L_c = \hbar/\sqrt{m_e k_B T}
\end{equation}

The transverse coherence is determined solely from the temperature of the electrons which is proportional to the temperature of the electron source and the ionisation energy.



uniform density ellipsoidal bunchs -> coulomb explosion reversal\cite{van_der_geer_simulated_2007}

bunch shaping \cite{mcculloch_arbitrarily_2011}


