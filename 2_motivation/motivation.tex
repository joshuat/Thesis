\chapter{Motivation for an optical dipole trap}

When extracting the ionised electrons from the atom cloud it is necessary to turn off the magnetic trapping fields and the magnetic fields of the Zeeman slower in order to prevent magnetic distortions of the electron bunch trajectory. This means that just before electron extraction the cloud is no longer trapped it begins to expand. The use of additional trapping mechanisms that do not interfere with the electron trajectories, such as an \gls{odt}, will help prevent this expansion. An \gls{odt} will also serve to stabilise the initial position of the electron bunches which tend to drift at present due to dispersal of currents, and hence magnetic fields, in the magnetic coils and their power supplies.

Using an \gls{odt} in this system should also allow an increase in the brightness and perhaps coherence of the source due to higher atom cloud densities during the ionisation process. For single-shot \gls{cdi} a combination of coherence and electron current is required in order for sufficient information to be captured in the resulting diffraction pattern.

\section{Coherence}
Unsurprisingly coherence is an important factor in \gls{cdi} since this technique relies on the interference of the diffracted waves. The transverse coherence length of a imaging beam must be approximately twice the width of the object being imaged\cite{spence_coherence_2004} in order to properly resolve structure.

For a quasi-homogeneous source\cite{nugent_coherent_2009}, the transverse coherence length $L_c$ can be related to the transverse momentum spread, and hence the temperature, through\cite{van_oudheusden_electron_2007}
\begin{equation}
L_c = \hbar/\sqrt{m_e k_B T}.
\end{equation}

The transverse coherence is determined solely from the temperature of the electrons which is proportional to the temperature of the electron source and the ionisation energy.

The coherence of the electron bunches could be increased with the sophisticated use of \glspl{odt} since the trap would serve to reduce the temperature of the atom cloud, and hence the electron bunches.

\section{Brightness}
Brightness is important in single-shot \gls{cdi} due to the need to maximise the scattered signal to ensure sufficient sampling of the specimen within the exposure time.

For the cold-atom source the transverse brightness at the source is given by\cite{reiser_theory_2008}
\begin{equation}
B_\perp = \frac{I_p m_e c^2}{4 \pi^2 \sigma_x \sigma_y k_B T},
\end{equation}
where $\sigma_x$ and $\sigma_y$ are the root mean squared source size along the respective axes, $I_p$ is the peak electron current and $T$ is the source temperature.

The brightness of an electron bunch can be increased by a reduction in the length of the bunch or by increasing the density. A short bunch is also necessary for ultrafast electron diffraction.

The use of an \gls{odt} will increase the density of the atom-cloud by trapping the otherwise expanding atom-cloud which will result in an increase of the electron bunch density and hence the brightness.
