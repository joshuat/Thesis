\chapter{Conclusion}

A crossed beam optical dipole trap was successfully created during this project. A fair portion of the atoms from the \gls{mot} can be trapped however a significant number of atoms are clearly escaping from the trapping region. A combination of a deeper \gls{odt} and a colder \gls{mot} should help to rectift this. The lifetime of the trap was measured to be $2.1\,\unit{ms}$. Hopefully with further work trap lifetimes in excess of $10\,\unit{ms}$ can be achieved. The size of the dipole trap is large enough to contain the excitation region.

\section{Further Work}

The first goal that is still to be achieved is to extract electrons from the \gls{odt} and hopefully provide a more stable electron beam. Due to recent technical difficulties with the accelerating structures this has not yet been done.

Trapping more atoms at colder temperatures will also improve the usefulness of the \gls{odt} within the \gls{caes}. This could be achieved with additional cooling of the \gls{mot} before loading the \gls{odt} and with a deeper trap. Significant amounts of litrature is availble on optimising the loading and operation of \gls{odt}.

Compression of the \gls{mot} and \gls{odt} should also improve the brightness of the \gls{caes} as with more atoms in the excitation region more electrons can be extracted.

The use of optical lattices to reduce disorder induced heating is another avenue worthy of exploration in attempts to reduce the temperature of the \gls{caes}.
