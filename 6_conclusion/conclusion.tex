\chapter{Conclusion}

The \gls{caes} aims to provide an alternative imaging source for use in ultra-fast, single-shot \gls{cdi} using electrons. This goal requires bright, femtosecond long bunches of extremely cold electrons to be produced. The source must also be stable and reliable.

In its current state the \gls{caes} is able to produce picosecond long extremely cold bunches of electrons. The brightness still needs to increase and the the temporal bunch length needs to be compressed. At the moment however the most limiting factor of the \gls{caes} is its instability. Due to remnant magnetic fields and instabilities in the atom cloud the trajectories of the electron bunches are erratic and unpredictable which makes reliable imaging next to impossible. The use of an \gls{odt} as discussed in this thesis aims to stabalise the source so that the electron bunches reliable travel along the same trajectories for every shot. Unfortunately, due to a technical issues with the \gls{caes}, it has not been possible to test the stability of the electron bunches when the \gls{odt} is used.

With the aforementioned technical issue soon to be solved the stability of the \gls{caes} using the \gls{odt} can be examined and if the electron signal is bright enough it should be possible to finally observe electron diffraction from samples.

If the electron bunches produced from the \gls{odt} are not bright or more brightness is desired (as it always is) then optimisation of the \gls{odt} be required. Reducing the temperature of the \gls{mot} will allow more of the atoms in the region of the dipole trap to become trapped. Once the atom temperature is low enough it will become beneficial to increase the size of the \gls{odt} at the expense of depth. Reducing the atom temperature can be achieved with any of a number of techniques such as Raman sideband cooling\cite{metcalf_laser_1999} and Sisyphus cooling\cite{metcalf_laser_1999}.


A crossed beam optical dipole trap was successfully created during this project. A fair portion of the atoms from the \gls{mot} can be trapped however a significant number of atoms are clearly escaping from the trapping region. A combination of a deeper \gls{odt} and a colder \gls{mot} should help to rectify this. The lifetime of the trap was measured to be $2.1\,\unit{ms}$. Hopefully with further work trap lifetimes in excess of $10\,\unit{ms}$ can be achieved. The size of the dipole trap is large enough to contain the excitation region.

\section{Further Work}

The first goal that is still to be achieved is to extract electrons from the \gls{odt} and hopefully provide a more stable electron beam. Due to recent technical difficulties with the accelerating structures this has not yet been done.

Trapping more atoms at colder temperatures will also improve the usefulness of the \gls{odt} within the \gls{caes}. This could be achieved with additional cooling of the \gls{mot} before loading the \gls{odt} and with a deeper trap. Significant amounts of litrature is availble on optimising the loading and operation of \gls{odt}.

Compression of the \gls{mot} and \gls{odt} should also improve the brightness of the \gls{caes} as with more atoms in the excitation region more electrons can be extracted.

The use of optical lattices to reduce disorder induced heating is another avenue worthy of exploration in attempts to reduce the temperature of the \gls{caes}.
