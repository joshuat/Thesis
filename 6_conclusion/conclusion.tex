\chapter{Conclusion}

The \gls{caes} aims to provide an alternative imaging source for use in ultra-fast, single-shot \gls{cdi} using electrons. This goal requires bright, femtosecond long bunches of extremely cold electrons to be produced. The source must also be stable and reliable.

In its current state the \gls{caes} is able to produce picosecond long extremely cold bunches of electrons. The brightness still needs to increase and the temporal bunch length needs to be compressed. At the moment however the most limiting factor of the \gls{caes} is its instability. Due to remnant magnetic fields and instabilities in the atom cloud the trajectories of the electron bunches are erratic and unpredictable which makes reliable imaging impossible. The use of an \gls{odt} as discussed in this thesis aims to stabilise the source so that the electron bunches reliably travel along the same trajectories for every shot. Tests of the electron stability with the \gls{odt} are currently underway.

With the aforementioned technical issue soon to be solved the stability of the \gls{caes} using the \gls{odt} can be examined and if the electron signal is bright enough it should be possible to finally observe electron diffraction from samples.

If the electron bunches produced from the \gls{odt} are not bright enough or more brightness is desired (as it always is) then optimisation of the \gls{odt} will be required. Reducing the temperature of the \gls{mot} will allow more of the atoms in the region of the dipole trap to become trapped. Once the atom temperature is low enough it will become beneficial to increase the size of the \gls{odt} at the expense of depth. Reducing the atom temperature can be achieved with any of a number of techniques such as Sisyphus cooling\cite{metcalf_laser_1999} and Raman sideband cooling\cite{lee_raman_1996, metcalf_laser_1999}. Another way to increase the number of atoms in the \gls{odt} is to increase the density of the atoms in the \gls{mot} via techniques such as magnetic compression. There are numerous other techniques and `tricks' that can be used to optimise both \glspl{mot} and \glspl{odt}\cite{kuppens_loading_2000, grimm_optical_2000}.

Currently the lifetime of the \gls{odt} is shorter than hoped. The techniques mentioned above combined with increasing the detuning may be enough to increase the lifetime sufficiently. Fortunately the lifetime of \glspl{odt} is something that has been studied in great detail by the \gls{bec} community\cite{barrett_all-optical_2001, arnold_all-optical_2011}.

The size of the atom cloud trapped by the \gls{odt} is greater than the size required ($500\,\unit{\mu m}$) for ultra-fast, electron diffraction with electrons of temperature $10\,\unit{K}$\cite{mcculloch_towards_2012}.

During the course of this project the crossed beam \gls{odt} was designed, constructed, integrated into the \gls{caes} and tested. Further progress can still be made by determining the scattering and trap depth `sweet spot' which is affected by the atom temperature, laser wavelength and laser beam waist. More power can also be extracted from the \gls{odt} laser system, via improvements to the \gls{ta} input coupling optics and isolator alignment.

